%%%%%%%%%%%%%%%%%%%%%%%%%%%%%%%%%%%%%%%%%%%%%%%%%%%%%%%%%%%%
%%% LIVECOMS ARTICLE TEMPLATE FOR BEST PRACTICES GUIDE
%%% ADAPTED FROM ELIFE ARTICLE TEMPLATE (8/10/2017)
%%%%%%%%%%%%%%%%%%%%%%%%%%%%%%%%%%%%%%%%%%%%%%%%%%%%%%%%%%%%
%%% PREAMBLE
\documentclass[9pt,bestpractices]{livecoms-new}
% Use the 'onehalfspacing' option for 1.5 line spacing
% Use the 'doublespacing' option for 2.0 line spacing
% Use the 'lineno' option for adding line numbers.
% The 'bestpractices' option for indicates that this is a best practices guide.
% Omit the bestpractices option to remove the marking as a LiveCoMS paper.
% Please note that these options may affect formatting.

\usepackage{lipsum} % Required to insert dummy text
\usepackage[version=4]{mhchem}
\usepackage{siunitx}
\DeclareSIUnit\Molar{M}
\usepackage[italic]{mathastext}
\graphicspath{{figures/}}

%%%%%%%%%%%%%%%%%%%%%%%%%%%%%%%%%%%%%%%%%%%%%%%%%%%%%%%%%%%%
%%% IMPORTANT USER CONFIGURATION
%%%%%%%%%%%%%%%%%%%%%%%%%%%%%%%%%%%%%%%%%%%%%%%%%%%%%%%%%%%%
\usepackage[colorinlistoftodos]{todonotes}
\newcommand{\versionnumber}{0.1}  % you should update the minor version number in preprints and major version number of submissions.
\newcommand{\githubrepository}{\url{https://github.com/michellab/alchemical-best-practices}}  %this should be the main github repository for this article

%%%%%%%%%%%%%%%%%%%%%%%%%%%%%%%%%%%%%%%%%%%%%%%%%%%%%%%%%%%%
%%% ARTICLE SETUP
%%%%%%%%%%%%%%%%%%%%%%%%%%%%%%%%%%%%%%%%%%%%%%%%%%%%%%%%%%%%
\title{Best Practices for Alchemical Free Energy Calculations : v\versionnumber}

\author[1*]{John D. Chodera}
\author[2*]{Antonia S. J. S. Mey}
\author[2*]{Julien Michel}
\author[3*]{David L. Mobley}
\author[4*]{Conor Parks}
\author[5*]{Julia E. Rice}
\author[6*]{Michael Shirts}
\affil[1]{Computational and Systems Biology Program, Sloan Kettering Institute, Memorial Sloan Kettering Cancer Center, New York NY 10065}
\affil[2]{Departments of Pharmaceutical Sciences and Chemistry, University of California, Irvine}
\affil[3]{EaStCHEM School of Chemistry, David Brewster road, Joseph Black Building, The King's Buildings, Edinburgh, EH9 3FJ, UK}
\affil[4]{Department of Chemistry, University of California, San Diego CA 92093}
\affil[5]{IBM Almaden Research Center, San Jose, CA 95120}
\affil[6]{Chemical and Biological Engineering, University of Colorado Boulder, Boulder, CO 80309}


\corr{john.chodera@choderalab.org}{JDC}
\corr{antonia.mey@ed.ac.uk}{ASJSM}
\corr{dmobley@mobleylab.org}{DLM}
\corr{info@julienmichel.net}{JM}
\corr{cdparks@ucsd.edu}{CP}
\corr{jrice@ibm.com}{JER}
\corr{michael.shirts@colorado.edu}{MS}

%\contrib[\authfn{1}]{These authors contributed equally to this work}
%\contrib[\authfn{2}]{These authors also contributed equally to this work}


\blurb{This LiveCoMS document is maintained online on GitHub at \githubrepository\\To provide feedback, suggestions, or help improve it, please visit the GitHub repository and participate via the issue tracker.}

%%%%%%%%%%%%%%%%%%%%%%%%%%%%%%%%%%%%%%%%%%%%%%%%%%%%%%%%%%%%
%%% ARTICLE START
%%%%%%%%%%%%%%%%%%%%%%%%%%%%%%%%%%%%%%%%%%%%%%%%%%%%%%%%%%%%

\begin{document}

\begin{frontmatter}
\maketitle

\begin{abstract}
%\todo[inline, color={green!20}]{ASJSM: @Volunteer write abstract}
%In your work, in this particular slot, please provide an abstract of no more than 250 words.
%Your abstract should explain the main contributions of your article, and should not contain any material that is not included in the main text.
%Please note that your abstract, plus the authorship material following it, must not extend beyond the title page or modifications to the LaTeX class will likely be needed.
Alchemical free energy calculations can be a useful tool for predicting free energy differences associated with the transfer of small molecules from one environment to another.
The hallmark of these methods is the use of modified potential energy functions to represent \emph{alchemical} intermediate states that cannot exist in chemistry; by analyzing simulation data collected from a series of bridging alchemical thermodynamic states, transfer free energies (or differences in transfer free energies) can be computed with orders of magnitude less simulation time than observing the process spontaneously. 
While these methods are highly flexible, care must be taken in avoiding common pitfalls to ensure that computed free energy differences can be robust and reproducible for the chosen forcefield, and that appropriate corrections are included to permit comparison with experimental data.
In this paper, we review current best practices for several popular application domains of alchemical free energy calculations, including relative and absolute small molecule binding free energy calculations to biomolecular targets.
\todo[inline]{JDC: What about biopolymer mutations?}

\end{abstract}

\end{frontmatter}



\todototoc
\listoftodos

%%%%%%%%%%%%%%%%%%%%
%              Introduction                  %
%%%%%%%%%%%%%%%%%%%%
\section{Introduction}
\label{sec:intro}
\todo[inline, color={green!20}]{ASJSM: @Volunteer write Intro}
%Meeting Members: 
%David Mobley, Conor Parks, Samarjeet, Julia Rice, Toni Mey, John Chodera, Michael Shirts
%[please add your name if you have contributed to the document!]
(%Original Members: John Chodera, Conor Parks, Heather Mayes)
%People to Invite: Clara Christ (drug discovery expertise), Hannes Loeffler (FESetup expertise)

Alchemical free energy calculations have become a mature technology for computing various properties related to the transfer of chemical species from one environment to another.
The domain of applicability for these calculations now involves such varied applications as the computation of protein-ligand binding free energies~\cite{binding-free-energies}, ligand selectivities~\cite{selectivity}, partition and distribution coefficients between different liquid phases (such as octanol-water partition coefficients~\cite{octanol-water-partition}), loss of affinity due to resistance mutations~\cite{resistance-mutations}, and changes in protein thermostability due to engineered mutations~\cite{thermostability}.

The defining characteristic of alchemical free energy calculations is the use of a series of modified potential functions $U(x; \lambda)$ in which an alchemical parameter $\lambda$ modulates interactions in a manner that cannot occur in real chemical systems.
One or more simulations are used to collect data from a multitude of alchemical states to compute a free energy difference between a chemical state ($\lambda_0$) and another chemical or alchemical reference state ($\lambda_1$),
\begin{eqnarray}
\Delta f &\equiv& f(\lambda_1) - f(\lambda_0) = - \ln \frac{Z(\lambda_1)}{Z(\lambda_0)}
\end{eqnarray}
where the dimensionless free energy $f(\lambda) \equiv \beta F(\lambda)$ is given in terms of partition functions $Z(\lambda)$,
\begin{eqnarray}
Z(\lambda) &=& \int dx \, e^{-u(x; \lambda)} .
\end{eqnarray}
Here, the inverse temperature $\beta \equiv (k_B T)^{-1}$ where $k_B$ is the Boltzmann constant, and the \emph{reduced potential} $u(x; \lambda)$~\cite{reduced-potential} is generally given by a trace over thermodynamic parameters with their conjugate dynamical variables,
\begin{eqnarray}
u(x;\lambda) &\equiv& \beta \left[ U(x;\lambda) + p \, V(x) + \sum_{i=1}^N \mu_i N_i(x) + \cdots \right]
\end{eqnarray}
where the collection of thermodynamic and alchemical parameters $\theta \equiv \{\beta, \lambda, p, \mu, \ldots\}$ defines a \emph{thermodynamic state}.
The physical transformation of interest may require several free energy differences $\Delta f$ to be computed in order to produce an estimate of the overall desired quantity.


%%%%%%%%%%%%%%%%%%%%
%              Prerequesites                %
%%%%%%%%%%%%%%%%%%%%
\section{Prerequisites}
\label{sec:pre}
%Here you would identify prerequisites/background knowledge that are assumed by your work and your checklist which you view as critical, ideally giving links to good sources on these topics.
%Checklists are normally focused on errors made by users with training and experience in molecular simulations, so you can assume a basic familiarity with the fundamentals of molecular simulations.
Readers are expected to have a good working knowledge of both biomolecular simulations and the theory underlying alchemical free energy calculations.
Good reviews of alchemical free energy calculation theory and practice can be found at {\color{red}[CITE REVIEWS]}.

\section{Scope and Goals}
\label{sec:scope}
\begin{itemize}
\item Preparation (focused on aspects unique to alchemical free energies), execution, and analysis of:
\begin{itemize}
\item Transfer free energies (hydration free energies, partition coefficients, etc.)
\item Binding free energies
\end{itemize}
\item Not in scope:
\begin{itemize}
\item Advanced ligand binding topics such as:
\item Covalent inhibition
\item Association which is not 1:1
\item Endpoint free energy methods
\item PMF binding free energy methods
\item Force field choice: Should be a separate document (review)
\end{itemize}
\end{itemize}

\textbf{Goals:}
\begin{itemize}
\item Checklist and background for new practitioners of relative and absolute alchemical free energy calculations: What should you pay attention to in setting up and running calculations in common codes and why.
\item Provide guidance to authors as to what should be reported about their protocols in a Methods section, either conforming to standard or reporting where they did not conform and why. Useful to reviewers as well.
\end{itemize}

\section{Checklist}
\label{sec:checklist}
\todo[inline, color={green!20}]{ASJSM: @Volunteer This needs to be revised and expanded, some initial thoughts were just thrown in.}
An attempt at identifying most important checklist items. 


% This provides a checklist which
% - spans a full page
% - consists of multiple sub-checklists
% - exists on a separate page
% This style of checklist will be especially helpful if you want to encourage readers to print and use your checklist in practice, as they
% can easily print it without also printing other material from your manuscript. However, other styles of checklist are also possible (below).
\begin{Checklists*}[p!]

\begin{checklist}{Step 0 -- Know what you want to simulate }
\textbf{What are the first questions that need addressing before setting up a molecular dynamics simulation}\\
Extensive explanation for the checklist questions can be found in section~\ref{sec:step0}.
\begin{itemize}
\item Can I get the required accuracy with the simulation I want to carry out
\item 
\item And finally
\end{itemize}
\end{checklist}

\begin{checklist}{Simulation preparation}
\textbf{How do I get started setting up an alchemical free energy calculation}
Extensive explanation for the checklist questions can be found in section~\ref{sec:step1}.
\begin{itemize}
\item Have I followed the Best practices for biomolecular simulation set up?
\item In a relative simulation, will I run into problems with clashing geometries in the ligand transformation or crystal waters?
\end{itemize}
\end{checklist}
\end{Checklists*}

\begin{Checklists*}[p!]
\begin{checklist}{Absolute simulations}
\textbf{What are the main things I need to consider for an absolute alchemical free energy calculation?}
Extensive explanation for the checklist questions can be found in section~\ref{sec:step2}.
\begin{itemize}
\item Topology
\item Restraints
\item Standard state handling
\end{itemize}
\end{checklist}

\begin{checklist}{Relative simulations}
\textbf{What are the main things I need to consider for an relative alchemical free energy calculation?}
Extensive explanation for the checklist questions can be found in section~\ref{sec:step2}.
\begin{itemize}
\item First thing
\item Also remember
\item And finally
\end{itemize}
\end{checklist}

\begin{checklist}{Analysis}
\textbf{This is all about analysis of the simulation}
Extensive explanation for the checklist questions can be found in section~\ref{sec:step4}.
\begin{itemize}
\item Are my simulations converged enough?
\item Am I using the right analysis techniques?
\end{itemize}
\end{checklist}

\end{Checklists*}
\clearpage

\section{Step 0 -- What can be expected from alchemical simulations?}
\label{sec:step0}
\begin{itemize}
\item What level of accuracy you can expect? 
\item What timescales and how many transformations can you address given available computational resources? 
\item Can you even hope to tackle the problem you are attempting? 
\end{itemize}

\section{Step 1 -- Simulation prerequesites}
\label{sec:step1}
\begin{enumerate}
\item Generate geometry of initial state: Reference biomolecular simulation preparation best practices
\item Relative: Generate geometry of final state (Mey)
\begin{itemize}
\item e.g. ligand ideally shouldn’t clash with receptor, etc.; satisfy constraints that might be imposed by protocol such as overlapping atoms, etc.
\item reference biomolecular preparation setup practices for placing ligand into binding site, but elaborate on constraints that must be considered for relative free energy calculations
\end{itemize}
\end{enumerate}

\section{Step 2 -- Simulation protocol selection}
\label{sec:step2}
\begin{enumerate}
\item Relative: Select a topology and produce an atom mapping for transformation (Mobley)
\begin{itemize}
\item Watching out for constraints to bonds for hydrogen: these cannot be allowed to change without including Jacobian terms
\item Share atoms between initial and final ligands if possible, otherwise colocalize pair of compounds and exclude their interactions with one another
\item Ring breaking/forming: Special are is needed; cite references
\item Tools: enlist Hannes Loeffler (developer of FESetup)?
\item Clarify terminology: Dual-topology, single-topology, etc.
\end{itemize}

\item Absolute: Identify \textbf{restraints and standard state handling}
\begin{itemize}
\item Practical use of Boresch restraints/other forms of restraints. How to choose atoms involved. etc. (e.g. see Heinzelmann/Gilson BRD4 work; Chodera also has stuff to add)
\item Levi Naden: Problems with analytical approximation to standard-state correction for Boresch restraints
\item Harmonic and flat-bottom restraints
\item Clarify terminology: Double decoupling, etc. See Feature Box below.
\end{itemize}
\item Identify whether \textbf{net charge is changing }and how this will be handled:
Issues that still need to be resolved: Charge corrections vs alchemically modify counterions

\item Select an \textbf{alchemical pathway}
\begin{enumerate}
\item Choice of alchemical Hamiltonian
	\begin{itemize}
	\item Softcore potentials are always recommended, but might not ALWAYS be necessary, e.g.:
	\item Pure changes in parameters of atoms that don’t insert/delete atoms (turn into dummies/from dummies)
	\item If roughly isosteric (e.g. lambda dynamics work from C. Brooks)
	\end{itemize}
\item Choice of discrete alchemical protocol (Shirts, Mey, Chodera)
	\begin{itemize}
	\item Many options: Adaptive scheme, Chebyshev polynomials, linear spacing, “choose your next lambda from data at this lambda”, optimal thermodynamic length approaches (separately: Shirts, Sivak, Huafeng Xu).
	\item Levi Naden had paper with lambda protocol which worked for all cases -- methane solvation, host-guest (including disappearing host)
	\end{itemize}
\item Relative
Three-stage protocol (discharge unique initial atoms, transform LJ, charge unique final atoms) vs softcore electrostatics/LJ

\item Absolute
	\begin{itemize}
	\item Select a \textbf{common alchemically-eliminated end state}
	\item Decoupled vs annihilated for electrostatics and LJ
	\item Sequential electrostatics and LJ versus simultaneous (recommend sequential)

\end{itemize}
\item Concerns:
Part of AMBER still can’t run at endpoints (lambda = 0 or 1); SANDER cannot but PMEMD can.

\end{enumerate}

\item Determine whether you need to \textbf{handle multiple binding modes}: (Mobley)
\begin{itemize}
\item In absolute: Confine-and-release, BLUES
\item In relative (ugh!!)
\end{itemize}
\item Determine \textbf{stopping conditions}
Uncertainty-directed stopping criteria can ensure target uncertainty is achieved


\item Select which \textbf{data should be saved and with which frequency}
\begin{itemize}
\item What data to save: dU/dlambda, Delta E’s between neighbor for BAR, between further for MBAR, …
\item BAR captures most of info with well-optimized lambda protocol, but MBAR when perhaps not, except when there are way too many lambda values.
\item Recommend against solely relying on TI when possible 
\item Recommend cross-comparing methods (TI (spline, trapezoid, etc.), BAR, MBAR) as diagnosis of trouble
\end{itemize}

\end{enumerate}

\section{Step 3 -- Overview of available analysis techniques}
\label{sec:step3}
\begin{enumerate}
\item Detecting boundary between equilibrated and production regions (Chodera: \url{http://dx.doi.org/10.1021/acs.jctc.5b00784})
\item Decorrelating samples for analysis
\begin{itemize}
\item Subsample different lambdas based on correlation times
\item Ensure all simulations at least 50x correlation time
\end{itemize}
\item Examining output data for common problems with discussions of what exactly to plot or look at; examples of typical curves for dV/dlambda and free energy versus lambda, for example
\begin{itemize}
\item Make sure ligand doesn’t tumble out of binding site (Mey has observed this)
\item Significant discrepancies between different free energy estimators (TI, BAR, MBAR)
\item Poor replica mixing (for replica-exchange)
\item Correlation time as a function of lambda as it would be expected to be a smooth
\item Dependence on initial conformation
\item Torsional analysis: Is it stuck in specific states? Only very rarely transitions?
\item More “usual suspects”
\end{itemize}
\item Estimators for free energies
\begin{itemize}
\item MBAR recommended if all energy differences are available
\item BAR just as good for highly optimized lambda values
\item TI should be roughly concordant, but quadrature error hard to quantify
\item Other variants useful in special circumstances (e.g. Z. Tan stochastic version)
\end{itemize}
\item Computing and reporting uncertainties on free energies
correlated bootstrap v. timeseries analysis
\item Other considerations for many transformations
Cycle closure error



\end{enumerate}

%%%%%%%%%%%%%%%%%%%%
%              Terminology                  %
%%%%%%%%%%%%%%%%%%%%
\section{Terminology and abbreviations}
\label{sec:tem-abbrev}
\begin{itemize}
\item Feature Box covering major technical terms and abbreviations
\item Examples:
\begin{itemize}
\item EXP, BAR, MBAR
\item Double decoupling, single-topology, dual-topology, hybrid-topology, coupled-topology
\item FEP (free energy peturbation), alchemical, AFE (alchemical free energy)
\end{itemize}
\end{itemize}
%%%%%%%%%%%%%%%%%%%%
%                Software                    %
%%%%%%%%%%%%%%%%%%%%
\section{Available software -- a summary}
\label{sec:software}
\begin{itemize}
\item Commercial:
   \begin{itemize}
    \item FEP+
    \end{itemize}
\item Free or low-cost for academics / commercial for industry:
	\begin{itemize}
	\item CHARMM / DOMDEC / CHARMM-OPENMM
	\item TIES and AMBER FEW? (Peter Coveney)
	\item AMBER / PMEMD
	\end{itemize}
\item Free (libre) open source:
	\begin{itemize}
	\item SIRE
	\item YANK
	\item gromacs 
	\item pmx for mutations
	\end{itemize}
\item Setup tools
	\begin{itemize}
	\item FESetup: AMBER, gromacs, Sire
	\item Lomap/Lomap2 : Relative alchemical transformation graph planning
	\end{itemize}
\item Analysis tools:
	\begin{itemize}
	\item Free Energy Workflows: Sire-specific free energy map analysis using weighted path averages
	\url{https://github.com/michellab/freenrgworkflows}
	\item Alchemlyb: Multipackage free energy analysis
	\url{https://github.com/alchemistry/alchemlyb}
	\item pymbar: MBAR implementation, but have to roll your own analysis wrapper
	\url{https://github.com/choderalab/pymbar}
	\end{itemize}
\end{itemize}

\section{Online resources}
\begin{itemize}
\item \url{http://www.ks.uiuc.edu/Training/Workshop/Urbana_2010A/lectures/TCBG-2010.pdf}
\item Basic Ingredients of Free Energy Calculations: A Review (\url{DOI: 10.1002/jcc.21450})
\item Good Practices in Free-Energy Calculations (\url{DOI: 10.1021/jp102971x})
\item Alchemical Free Energy Methods for Drug Discovery: Progress and Challenges (\url{doi: 10.1016/j.sbi.2011.01.011})
\item Alchemistry wiki: \url{http://www.alchemistry.org/wiki/Best_Practices}
\end{itemize}

\section*{Author Contributions}
%%%%%%%%%%%%%%%%
% This section mustt describe the actual contributions of
% author. Since this is an electronic-only journal, there is
% no length limit when you describe the authors' contributions,
% so we recommend describing what they actually did rather than
% simply categorizing them in a small number of
% predefined roles as might be done in other journals.
%
% See the policies ``Policies on Authorship'' section of https://livecoms.github.io
% for more information on deciding on authorship and author order.
%%%%%%%%%%%%%%%%

(Explain the contributions of the different authors here)

% We suggest you preserve this comment:
For a more detailed description of author contributions,
see the GitHub issue tracking and changelog at \githubrepository.

\section*{Other Contributions}
%%%%%%%%%%%%%%%
% You should include all people who have filed issues that were
% accepted into the paper, or that upon discussion altered what was in the paper.
% Multiple significant contributions might mean that the contributor
% should be moved to authorship at the discretion of the a
%
% See the policies ``Policies on Authorship'' section of https://livecoms.github.io for
% more information on deciding on authorship and author order.
%%%%%%%%%%%%%%%

(Explain the contributions of any non-author contributors here)
% We suggest you preserve this comment:
For a more detailed description of contributions from the community and others, see the GitHub issue tracking and changelog at \githubrepository.

\section*{Potentially Conflicting Interests}
%%%%%%%
%Declare any potentially competing interests, financial or otherwise
%%%%%%%

Declare any potentially conflicting interests here, whether or not they pose an actual conflict in your view.

\section*{Funding Information}
%%%%%%%
% Authors should acknowledge funding sources here. Reference specific grants.
%%%%%%%
FMS acknowledges the support of NSF grant CHE-1111111.

\bibliography{livecoms-sample}

%%%%%%%%%%%%%%%%%%%%%%%%%%%%%%%%%%%%%%%%%%%%%%%%%%%%%%%%%%%%
%%% APPENDICES
%%%%%%%%%%%%%%%%%%%%%%%%%%%%%%%%%%%%%%%%%%%%%%%%%%%%%%%%%%%%

%\appendix


\end{document}
